\section{はじめに}
このセクションではLinux Kernelに対する知見をカーネルモジュール制作を通して深めることを目的としています。
単純なカーネルモジュールを制作してもあまり意味が無いので、ここではカーネルモジュールとして動作するEchoサーバを作ることを題材にします。

\subsection{カーネルモジュールとは}
カーネルモジュールとは、Linux上で動的につまりは起動中でも追加削除可能なモジュールを指す。
通常のOSではカーネルにモジュールを追加するとカーネルそのものを再構築する必要がでてくるが、Linuxカーネルモジュールはそれを必要とせずにモジュールの追加、利用、削除が可能となっている。
ここで紹介するカーネルモジュールはその特性上、ローダブルカーネルモジュールと呼ばれることがある。
現在ロードされているカーネルモジュールを確かめるには\mintinline{bash}{lsmod}で確認することができる。

\subsection{カーネルからHello,World!!}

\subsection{}
